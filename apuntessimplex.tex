\documentclass{article}

\usepackage[spanish]{babel}
\usepackage{amsmath}
\usepackage[utf8]{inputenc}
\title{Apuntes de Método Simplex}
\author{Froylan Gonzalez Gomez}

\begin{document}

\maketitle

\section{Introducción}

El método simplex es un algoritmo para resolver problemas de
Programación Lineal. Fue inventado por George Dantzig en 1947.

\section{Ejemplo}
Ilustraremos la aplicación del método simplex con un ejemplo:

Resuelve el siguiente problema:

\begin{equation}
  
  \begin{aligned}
    \text{Maximizar} \quad & 3x_1+x_2 \\
    \text{Sujeto a} \quad &
    
    \begin{aligned}
      2x_2+3x_1 & \leq 10 \\
      5x_1+x_2 & \leq 8 \\
      x_1,x_2  \geq 0

    \end{aligned}
  \end{aligned}
\end{equation}

Agregamos las variables de holgura para pasar el problema a su forma
simplex

\begin{equation}
  
  \begin{aligned}
    \text{Maximizar} \quad & 3x_1+x_2 \\
    \text{Sujeto a} \quad &
    
    \begin{aligned}
      2x_1+3x_2+x_3 10 \\
      5x_1+x_2+x_4 8 \\
      x_1,x_2  \geq 0

    \end{aligned}
  \end{aligned}
\end{equation}

    
A continuación obtenemos un \emph{tablero simplex} despejando las
variables de holgura     
    



\end{document}

